\documentclass[10pt,a4paper]{book}
\usepackage[utf8]{inputenc}
\usepackage[russian]{babel}
\usepackage[OT1]{fontenc}
\usepackage{xcolor}


\title{DnD book}
\author{darling}
\date{\today}

\newcommand{\TODO}[1]{\textcolor{red}{TODO: #1}}

\newcommand{\monster}[1]{\textit{#1}}
\newcommand{\dm}[1]{\\\textcolor{orange}{\textit{#1}}\\}
\renewcommand{\check}[2]{\textit{#1 со Сл.#2}}

\renewcommand{\d}[1]{\textbf {d{#1}}}


\begin{document}
	\maketitle
	\tableofcontents
	\newpage
	\chapter{Для хозяина подземелья}
		\section{Контакты}
			GitHub: https://github.com/69darling69
		
		\section{Термины}
			Вот основные термины, которые могут использоваться дальше по тексту. Если вы встретили незнакомое для вас слово или сокращение, то вернитесь на эту страницу и найдите его. Если ничего подобного вы не нашли, то сообщите автору для добавления, а также поищите самоястоятельно в интернете.
			\begin{itemize}
				\item D\&D (Dungeons and Dragons) - "Подземелья и Драконы". Настольно-ролевая игра.
				\item 5e (5 edition) - "Пятая редакция". Версия D\&D, всего существует 1, 2, 3, 3.5, 4, 5.
				\item \TODO{Добавить другие термины}
			\end{itemize}
	
		\section{Предисловие}
		
			Данный модуль явлсяется адаптацией одной из компьюетрных игр. Он предназначени для пятой редакции D\&D, однако возможно использовать и с другими редакциями, оставив только завязку и сюжет. Это первое приключение автора, поэтому любые предложения и критика приветствуются. Для этого оставляйте свои issue на GitHub или предлагайте Pull Request.
			\\\\
			Для проведения игр вам понадобится:
			\begin{itemize}
				\item Базовые правила D\&D 5e (для углубленного понимания механик игры, автор все же советует изучить PHB)
				\item Данная книга
				\item Кубики
					\begin{itemize}
						\item \d{4}
						\item \d{6}
						\item \d{8}
						\item \d{10}
						\item \d{12}
						\item \d{20}
						\item \d{100} (для удобства используется второй \d{10} с десятками)
					\end{itemize}
				\item 4 игрока (хотя возможно проводить игры для любого количества игроков, но все же самым идеальным вариантом будет именно 4, так как приключения написано из этого расчета)
				\item \TODO{Нужно ли что-то еще для игры?}
			\end{itemize}
			Теперь же, когда все подготовки закончены, удачи вам и вашим героям в этом приключении.\TODO{Нужна какая-то фраза, которая будет потом по игре. Добавить ее сюда. Что-то типа "счастья великому королю".}	
		
	\chapter{Введение}
		\TODO{Написать введения}
		\section{Общая информация}
			\TODO{Написать общую инфомрацию}
		\section{Обзор приключения}
			\TODO{Написать обзор приключения}
	
	\chapter{Выжить любой ценой}
		Деревня Золотые рудники располагается посреди огромной поляны, опаясанной лесом с севера-запада и рекой с юга. Деревня была основана \TODO{Имя основателя}, чей памятник стоит на главной площади. Он был простым фермером, потомки которого до сих пор управляют данным местом, хотя и не так хорошо. Их уже давно никто не считают главенствующими, а эту роль на данный момент занимают три семьи: \TODO{Фамилии семей, их описание}. На ближайшие десятки километров нет ни единого поселения. Это связано с распространающимися болотами, которые способствуют вымиранию почвы, вследствии чего земледелие невозможно как таковое. Однако Золотые рудники находятся в так называемом "Шаре плодородия". По какой-то причине болото не подходит к деревне ближе, чем на километр. Поэтому земледелие в деревне продолжается своим ходом. Торговли как таковой тут не существует, так как пути для передвижения караванов слишком опасны. Однако в городе есть торговец \TODO{Имя торговца}, который каждое воскресное утро выходит на рынок с новыми товарами. Помимо него, на продажу выставляют свои товары местный кузнец \TODO{Имя кузнеца}, два брата \TODO{Имена братьев} - торговцы всякой всячины, а также несколько мелких фермеров, продающие жизненно необхожимые продукты питания.

		Однако персонажи начнут свое приключение не из этой точки. Их встреча состоится, вернее уже состоялась, прямо в гуще болота. Они просыпаются в доме ведьмы, связанные и абсолютно беспамятные. Вы можете изменить данную предысторию, если хотите более тесных взаимоотношений между персонажами. Для этого замените эту главу собственной завязкой и переходите сразу к следующей главе\TODO{Ссылка на следующую главу}.
		
		\section{Пробуждение}
			Когда персонажи приходят в себя, они находят себя связанными посреди небольшого домика. Дом предствялет из себя квадрат со стороной 20 футов\TODO{Картинка дома ведьмы}. Ровно посередине дома находится колонна, которая держит крышу, сделанную из лишайника и мха. Именно к этой колонне и привязаны персонажи. Все они сидят спиной друг к другу и видят общую область домика. В домике присутствует запах пресени, а также еще какой-то непонятный запах, который идет из котла. Возле котла стоит \monster{Ведьма}, которая кладет в котел какие-то ингридиенты и приговаривает себе что-то под но.
			\dm{Вы просыпаетесь от ужасного запаха плесени, который льется вам прямо в нос. Открыв глаза, вы видите старушку, которая что-то варит в большом котелке. Она подкидывает туда какие-то ингридиенты и что-то приговаривает себе под нос.}
			Персонажи могут осмотреться, однако если они вызовут какой-то шум, то \monster{Ведьма} это услышит, если пройдет проверку \check{Внимательность}{10}
		
\end{document}
