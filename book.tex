\documentclass[10pt,a4paper]{book}
\usepackage[utf8]{inputenc}
\usepackage[russian]{babel}
\usepackage[OT1]{fontenc}
\usepackage{xcolor}


\title{DnD book}
\author{darling}
\date{\today}

\newcommand{\TODO}[1]{\textcolor{red}{TODO: #1}}

\newcommand{\monster}[1]{\textit{#1}}
\newcommand{\dm}[1]{\\\textcolor{orange}{\textit{#1}}\\}
\renewcommand{\check}[2]{\textit{#1 со Сл.#2}}
\newcommand{\thing}[1]{\textit{#1}}
\newcommand{\hardcore}[1]{\\\textcolor{red}{\textbf{#1}}\\}

\renewcommand{\d}[2][1]{\textbf {#1d{#2}}}


\begin{document}
	\maketitle
	\tableofcontents
	\newpage
	\chapter{Для хозяина подземелья}
		\section{Контакты}
			GitHub: https://github.com/69darling69
		
		\section{Термины}
			Вот основные термины, которые могут использоваться дальше по тексту. Если вы встретили незнакомое для вас слово или сокращение, то вернитесь на эту страницу и найдите его. Если ничего подобного вы не нашли, то сообщите автору для добавления, а также поищите самоястоятельно в интернете.
			\begin{itemize}
				\item D\&D (Dungeons and Dragons) - "Подземелья и Драконы". Настольно-ролевая игра.
				\item 5e (5 edition) - "Пятая редакция". Версия D\&D, всего существует 1, 2, 3, 3.5, 4, 5.
				\item \TODO{Добавить другие термины}
			\end{itemize}
	
		\section{Предисловие}
		
			Данный модуль явлсяется адаптацией одной из известнейших компьюетрных игр. Он предназначени для пятой редакции D\&D, однако возможно использовать и с другими редакциями, оставив только завязку и сюжет. Приключение задумано для персонажей 1 уровня, однако возможно играть и более высокими уровнями, однако тогда требуется повышение сложности для првоерок и столкновений. Это первое приключение автора, поэтому любые предложения и критика приветствуются. Для этого оставляйте свои issue на GitHub или предлагайте Pull Request.
			\\\\
			Для проведения игр вам понадобится:
			\begin{itemize}
				\item Базовые правила D\&D 5e (для углубленного понимания механик игры, автор все же советует изучить PHB)
				\item Данная книга
				\item Кубики
					\begin{itemize}
						\item \d{4}
						\item \d{6}
						\item \d{8}
						\item \d{10}
						\item \d{12}
						\item \d{20}
						\item \d{100} (для удобства используется второй \d{10} с десятками)
					\end{itemize}
				\item 4 игрока (хотя возможно проводить игры для любого количества игроков, но все же самым идеальным вариантом будет именно 4, так как приключения написано из этого расчета)
				\item \TODO{Нужно ли что-то еще для игры?}
			\end{itemize}
			В некоторых местах, где сделаны некоторые допущения по правилам или же сложности есть текст, помечанный красным цветом. Это для так называемого режима hardcore, который отклоняется от повествовательного стиля истории и заставляет игроков чувствовать дыхание смерти за каждым углом. Такой режим с одной стороны может погубить не одного персонажа, однако выжившие прочувствуют историю самым сердцем. Автор не советует использовать это как наказание игроков, ведь вы играете не протви них, а с ними. Никогда это не забывайте.
			Теперь же, когда все подготовки закончены, удачи вам и вашим героям в этом приключении.\TODO{Нужна какая-то фраза, которая будет потом по игре. Добавить ее сюда. Что-то типа "счастья великому королю".}	
		
	\chapter{Введение}
		\TODO{Написать введения}
		\section{Общая информация}
			\TODO{Написать общую инфомрацию}
		\section{Обзор приключения}
			\TODO{Написать обзор приключения}
	
	\chapter{Выжить любой ценой}
		Деревня Золотые рудники располагается посреди огромной поляны, опаясанной лесом с севера-запада и рекой с юга. Деревня была основана \TODO{Имя основателя}, чей памятник стоит на главной площади. Он был простым фермером, потомки которого до сих пор управляют данным местом, хотя и не так хорошо. Их уже давно никто не считают главенствующими, а эту роль на данный момент занимают три семьи: \TODO{Фамилии семей, их описание}. На ближайшие десятки километров нет ни единого поселения. Это связано с распространающимися болотами, которые способствуют вымиранию почвы, вследствии чего земледелие невозможно как таковое. Однако Золотые рудники находятся в так называемом "Шаре плодородия". По какой-то причине болото не подходит к деревне ближе, чем на километр. Поэтому земледелие в деревне продолжается своим ходом. Торговли как таковой тут не существует, так как пути для передвижения караванов слишком опасны. Однако в городе есть торговец \TODO{Имя торговца}, который каждое воскресное утро выходит на рынок с новыми товарами. Помимо него, на продажу выставляют свои товары местный кузнец \TODO{Имя кузнеца}, два брата \TODO{Имена братьев} - торговцы всякой всячины, а также несколько мелких фермеров, продающие жизненно необхожимые продукты питания.

		Однако персонажи начнут свое приключение не из этой точки. Их встреча состоится, вернее уже состоялась, прямо в гуще болота. Они просыпаются в доме ведьмы, связанные и абсолютно беспамятные. Вы можете изменить данную предысторию, если хотите более тесных взаимоотношений между персонажами. Для этого замените эту главу собственной завязкой и переходите сразу к следующей главе\TODO{Ссылка на следующую главу}.
		
		\section{Пробуждение}
			Когда персонажи приходят в себя, они находят себя связанными посреди небольшого домика. Дом предствялет из себя квадрат со стороной 20 футов\TODO{Картинка дома ведьмы}. Ровно посередине дома находится колонна, которая держит крышу, сделанную из лишайника и мха. Именно к этой колонне и привязаны персонажи. Все они сидят спиной друг к другу и видят общую область домика. В домике присутствует запах пресени, а также еще какой-то непонятный запах, который идет из котла. Возле котла стоит \monster{Ведьма}, которая кладет в котел какие-то ингридиенты и приговаривает себе что-то под нос. \TODO{Описание ведьмы}
			\dm{Вы просыпаетесь от ужасного запаха плесени, который льется вам прямо в нос. Открыв глаза, вы видите старушку, которая что-то варит в большом котелке. Она подкидывает туда какие-то ингридиенты и что-то приговаривает себе под нос.}
			Персонажи могут осмотреться, однако если они вызовут какой-то шум, то \monster{Ведьма} это услышит, если пройдет проверку \check{Внимательность}{10}. Услышав что-то она застынет на несколько секунд, после чего медленно повернет свой взгляд в сторону персонажей. Один из ее глаз косит. В руках она держит какой-то красный гриб с белыми кружками. Персонаж, прошедший проверку \check{Природа}{10} поймет, что это так называемый красный гриб, который используется для приготовления некоторых блюд.
			\dm{Уже проснулись, голубчики? А я как-раз почти закончила, осталось добавить лишь паучий глаз. Вроде у меня оставалось парочку... Или нет. Что же, придется сходить за ним. Я скоро приду, никуда не дергайтесь. Хи-хи...}
			С нервным смешком она хватает свою сумку, выхватывает ключ из замочной скажины двери закрывает дверь, выйдя на улицу. Спустя несколько секунд слышен щелчок закрывающейся двери и скрип дерева. Все вокруг затихает, лишь изредка из котла доносится бурление.
			\monster{Ведьма} ушла примерно на 20 игровых минут. Этого времени будет достаточно, чтобы уйти до ее возвращения, однако если вы почувствуете, что выши игроки слишком затянули с побегом, то подгоните их. Узнайте у игроков их \textit{Пассивную внимательность} (если не сделали этого заранее) и сообщите игроку с самым большим значением, что он слышит смешок, который доносится из окна. Выглянув из домика, вдалеке можно заметить \monster{Ведьму}, которая возвращается назад.
			Существует несколько способов освободиться, вот самые очевидные:
			\begin{itemize}
				\item Пройдя проверку \check{Атлетики}{10} персонаж может порвать веревку.
				\item Персонаж может использовать заклинание с только вербальным компонентом для сотворения
			\end{itemize}
			Если ваш игрок нашел более креативный способ как выбраться, не поскупитесь и дайте ему благославление.
			\TODO{Осмотр домика ведьмы}
			\subsection{Игра на выживание}
			Если по какой-то причине игроки не смогли выбраться до возвращения \monster{Ведьмы}, то вернувшись она доварит зелья, параллельно приговаривая какой прекрасный у нее получается отвар. После чего она развяжет персонажей и выпустит их наружу. На вопросы она будет отвечать размыто, каждый раз говоря, что судьба все решит за них. Она выпустит персонажей на улицу, после чего захлопнет дверь.
			\dm{Выйдя на улицу, вы надеялись хоть на немного забыть о том ужасном запахе сырости, который наполняет весь дом, однако на болоте, где вы сейчас оказались, запах сырости присутствует везде. Вокруг видны лишь редкие деревья, наполняющие рваную землю. Вы слышите у себя за спиной звук захлопывающейся двери. Спустя всего несколько секунд дверь снова открывается и оотуда слышится голос ведьмы "Чуть не забыла, вам подарочек от меня". Вы видите как из двери вылетает пузырек с зеленоватой жидкостью, который прилетает прямо в одного из вас. Все кто рядом также попадает под брызги этой жижы. "А теперь бегите, малыши". С таким криком вновь захопывается дверь.}
			Ведьма кидает в персонажей \thing{Взрывное зелье отравления}. На самом деле это зелье наносит \d{4} единиц урона каждый ход в течении трех ходов, однако такое количество урона, скорее всего, убьет персонажей. Узнайте до игры сколько у каждого персонажа здоровья. Если какой-то персонаж способен выдержать 12 урона и не умереть, то пусть ведьма кинет зелье именно в него. Он будет получать \d{4} урона в течении трех ходов. А остальные же будут получать всего по 1 урону за ход, так как зелье попало на них лишь слегка.
			\hardcore{Пусть каждый персонаж получает \d{4} урона каждый ход в течени  трех ходов. Это будет наказание за неудачный побег. Скорее всего некоторые персонажи будут вынуждены кидать \textit{Спасброски смерти}. Если кто-то выживет, то это станет сильной мотивацией, чтобы отомстить \monster{Ведьме}.}
		
\end{document}
